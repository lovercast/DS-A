\documentclass[11pt]{article}
\usepackage{fullpage}
\title{Template}
\author{Name}
\begin{document}
\maketitle

\section{section}
\subsection*{subsection without number}
text \textbf{bold text} text. Some math: $2+2=5$
\subsection{subsection}
text \emph{emphasized text} text. \cite{WC:1953}
discovered the structure of DNA.

A table:
\begin{table}[!th]
\begin{tabular}{|l|c|r|}
\hline
first & row & data \\
second & row & data \\ 
\hline
\end{tabular}
\caption{This is the caption}
\label{ex:table}
\end{table}

The table is numbered \ref{ex:table}.


\end{document}

\begin{comment}
Latex cheat sheet
Document Classes
book
report
article
letter
slides
\documentclass{class}, \begin{document} ... \end{document}

Common documentclass options
10pt/11pt/12pt
letterpaper/a4paper
twocolumn
twoside
landscape
draft [double space lines]

Packages
before \begin{document} use \usepackage{package}
fullpage
anysize - \marginsize{l}{r}{t}{b}
multicol - \begin{multicols}{n}
latexsym - use latex symbol font
graphicx - \includegraphics[width=x]{file}
url - \url{http://...}

Title
Usage: before \begin{document}. The decalration \maketitle goes at the top of the document
\author{text}
\title{text}
\date{text}

Miscellaneous
\pagestyle{empty}
\tableofcontents

Document Structure
\part{title}    subsubsection{title}
\chapter{title} \paragraph{title}
\section{title} \subparagraph{title}
\subsection{title}
Use a * as in \section*{title} to not number a particular item -
these items will not have entries in the table of contents.

Text environments
\begin{comment}
\begin{quote}
\begin{quotation} 
\begin{verse}

Lists
\begin{enumerate}
itemize
description
\item
\item[x] use item instead of normal bullet or number -- required for descriptions

References
\label{marker} Set a marker for cross-reference, 
    often of the form \label{sec:item}.
\ref{marker}
\pageref{marker}
\footnote{text}

Floating bodies
\begin{table}[place] add numbered table
\begin{figure}]place] add numbered figure
\begin{equation}[place]
\captions{text}
The place is a list valid placements for the body. t=top, h=here, b=bottom, p=separate page, !=place even if ugly

Text properties
FontFace
command         declaration         
textrm          rmfamily
textsf          sf
texttt          tt
textmd          md
textbf          bf
textup          upshape
textit          itshape
textsl          slshape
textsc          scshape
emph            em
textnormal      normalfont
underline

Fonst Size
\tiny
\scriptsize
\footnotesize
\small
\normalsize
\large
\Large
\LARGE
\huge
\Huge

Verbatim Text
\begin{verbatim}
\begin{verbatim*}
\verb!text!

Justification
\begin{center}      \centering
\begin{flushleft}   \raggedright
\begin{flushright}  \raggedleft

Miscellaneous
\linespread{x}      changes the line spacing by the multiplier x.

Text-mode symbols
Symbols
\&      \_      \ldots      \textbullet
\$      \^{}    \textbar    \textbackslash
\%      \#      \S
\0
hyphen -
en-dash --
em-dash ---

Line and page breaks
\\               begin a new line without new paragraph
\\*              prohibi pagebreak after linebreak.
\kill           don't print current line
\pagebreak      start new page
\noindent       do not indent current line

Misc
\today
$sim$
\@.             indicate that the . ends a sentence after following an upper case letter
\hspace{l}      horizontal space of length l (ex l = 20pt).
\vspace{l}      Vertical space of length l.
\rule{w}{h}     line of width w and height h.

Tabular environments
tabbing environment
\= set tab stop     \> go to tab stop

tabular environment
\begin{array}[pos]{cols}
\begin{tabular}[pos]{cols}
\begin{tabular*}{width}[pos]{cols}

tabular column specification
l               left-justified
c               centred
r               right justified
p{width}        same as \parbox[t]{width}
@{decl}         inset decl instead of inter-column space.
|               insets a verical line between columns.

Example
\begin{tabular}{| l | c | r |}
    \hline
    1 & 2 & 3 \\
    \hline
    4 & 5 & 6\\
    \hline
    7 & 8 & 9 \\
    \hline
\end{tabular}

tabular elements
\hline
\cline{x-y} horizontal line across columns x through y
\multicolumn{n}{cols}{text}
    a cell that spancs n columns, with ocls column specification.

MathMode
for inline math, use \(...\) or $ ... $. For displayed math,
use \[...\] or \begin{equation}.
superscript ^{x} subscript _{x}
\frac{x}{y}     \sum_{k=1}^n
\sqrt[n]{x}     \prod_{k=1}^n

Math mode symbols
\leq    \geq    \neq    \approx 
\times      \div    \pm     \cdot   
^{\circ}    \circ   \prime  \cdots
infty   neg     wedge   vee
supset  forall  in  rightarrow
subset  exists  notin   Rightarrow
cup     cap     min     Leftrightarrow
dot a   hat a   bar a   tilde a 

Bibliography and Citations
When using Bibtex, you need to run latex, bibtex and latex twice more to resolce dependencies.

Citation types
\cite{key}
\citeA{key}
\citeN{key}
\shortcite{key}
\shortciteA{key}
\shortciteN{key}
\citeyear{key}

All the above have an NP variant without parentheses eg. \citeNP

BIBTEX entry types
@article
@book
@booklet
@conference
@inbook
@incollection
@misc
@phdthesis
@proceedings
@techreport
@unpublished

Bibtex fields
address
author
booktitle
chapter
edition
editor
institution
journal
key
month
note
number
organization
pages
publisher
school
series
title
type
volume
year

Common Bibtex style files
abbrv   Standard    abstract    alpha with abstract
alpha   Standard    apa     APA
plain   Standard    unsrt   unsorted

The latex document should have the following two lines just
before the \end{document}, where bibfile.bib is the name of the Bibtex file.

\bibliographystyle{plain}
\bibliography{bibfile}

Bibtex example
The bibtex database goes in a file called file.bib, which is processed with 
bibtex file.

@String{N = {Na\-ture}}
@Article{WC:1953,
author  = {James Watson and Francis Crick},
title   = {A structure for Deoxyribose Nucleic Acid},
journal = N,
volume  = {171},
pages   = {737},
year    = 1953
}

Sample Latex document
\documentclass[11pt]{article}
\usepackage{fullpage}
\title{Template}
\author{Name}
\begin{document}
\maketitle

\section{section}
\subsection*{subsection without number}
text \textbf{bold text} text. Some math: $2+2=5$
\subsection{subsection}
text \emph{emphasized text} text. \cite{WC:1953}
discovered the structure of DNA.

A table:
\begin{table}[!th]
\begin{tabular}{|l|c|r|}
\hline
first & row & data \\
second & row & data \\ 
\hline
\end{tabular}
\caption{This is the caption}
\label{ex:table}
\end{table}

The table is numbered \ref{ex:table}.
\bibliographystyle{plain}
\bibliography{file.bib}
\end{document}



\end{comment}
